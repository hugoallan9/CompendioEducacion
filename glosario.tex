

\INEchaptercarta{Glosario}{}




\begin{description}
\item[Alfabetismo: ] Cualidad o estado de las personas que saben leer y escribir, se calcula en base a las personas mayores de catorce años.
\item[Año académico: ] Período anual de enseñanza y evaluación durante el cual los estudiantes asisten a clases o rinden exámenes en forma continua, sin considerar interrupciones breves. Su duración puede ser inferior a 12 meses, aunque habitualmente no menor a 9 meses. Dentro de un país, el año académico puede tener distinta duración según el nivel de educación o el tipo de establecimiento. También se conoce como ‘año escolar’ o ‘año lectivo’, principalmente en los niveles inferiores al nivel terciario o de educación superior.

\item[Área rural:   ] Se definen a los lugares poblados que se reconocen oficialmente con la categoría de aldeas, caseríos, parajes, fincas, etc., de cada municipio. Incluye a la población dispersa, según Acuerdo Gubernativo del 7 de abril de 1938. 
\item[Área urbana:  ] Se consideró como área urbana a las ciudades, villas y pueblos (cabeceras departamentales y municipales), así como a aquellos otros lugares poblados que tienen la categoría de colonia o condominio y los mayores de 2,000 habitantes, siempre que en dichos lugares, el 51 por ciento o más de los hogares disponga de alumbrado con energía eléctrica y de agua por tubería (chorro) dentro de sus locales de habitación (viviendas). Al igual que los censos anteriores, se incluyó como área urbana todo el municipio de Guatemala. 
\item[Asistencia médica: ] Conjunto de servicios que se proporcionan al individuo a través de personal calificado en el área médica, con el fin de promover, proteger y restaurar la salud.
\item[Campo de estudio o campos científicos: ] Los campos de estudio son agrupaciones basadas en las ciencias que congregan cada disciplina de educación superior. El Manual de Frascati ofrece las siguientes agrupaciones: ciencias naturales, ingeniería y tecnologías, ciencias médicas, ciencias agrícolas, ciencias sociales y humanidades.
\item[Educación formal: ] Educación institucionalizada, intencionada y planificada por organizaciones públicas y organismos privados acreditados. En su conjunto, esta constituye el sistema educativo formal del país. Por consiguiente, los programas de educación formal son reconocidos por las autoridades nacionales pertinentes o instancias equivalentes, por ejemplo, cualquier otra institución que colabore con las autoridades nacionales o sub nacionales de educación. La educación formal comprende esencialmente la educación previa al ingreso al mercado laboral. Con frecuencia, la educación vocacional, la educación para necesidades especiales y parte de la educación de adultos se reconocen como parte integral del sistema nacional de educación formal.
\item[Educación Superior o Educación terciaria: ] Es la que se desarrolla sobre la base de los conocimientos adquiridos en la educación secundaria, proporcionando actividades educativas y de aprendizaje en campos especializados de estudio. Se caracteriza por promover el aprendizaje a un nivel elevado de complejidad y especialización. Si bien la educación terciaria incluye lo que es comúnmente entendido como “educación académica”, pero incluye también la educación vocacional o profesional avanzada.
\item[Facultad o Escuela Universitaria: ] Centro docente donde se imparten estudios superiores especializados en alguna materia o rama del saber. Generalmente constituye una subdivisión de una universidad. Las facultades deben su nombre al hecho de que poseen la atribución o potestad legalmente reconocida de otorgar grados académicos, lo que supone que se las considera autoridades calificadas para certificar la calidad de la formación y los conocimientos de sus propios egresados. Una facultad universitaria puede otorgar todo tipo de grados académicos, tanto de pregrado como de postgrado. No existe ninguna diferencia entre una facultad universitaria y una escuela universitaria, aunque las escuelas están históricamente vinculadas a las ingenierías y a las carreras científicas.
\item[Gasto de Gobierno: ] Gasto de Gobierno o gasto público es la cantidad de recursos financieros, materiales y humanos que el sector público representado por el gobierno emplea para el cumplimiento de sus funciones, entre las que se encuentran de manera primordial la de satisfacer los servicios públicos de la sociedad. Así mismo el gasto público es un instrumento importante de la política económica de cualquier país pues por medio de este, el gobierno influye en los niveles de consumo, inversión, empleo, etc. Así, el gasto público es considerado la devolución a la sociedad de algunos recursos económicos que el gobierno captó vía ingresos públicos, por medio de su sistema tributario.
\item[Graduación: ] La conclusión exitosa de un programa educativo. Es factible que un graduado tenga más de una graduación (incluso durante el mismo año académico) si el estudiante estuvo matriculado en dos o más programas simultáneamente y los terminó exitosamente.
\item[Grupo étnico: ] Se refiere al derecho individual a la auto identificación de la persona como indígena o no indígena. La respuesta se obtiene por medio de pregunta directa y no por simple observación.
\item[Hogar: 	 ] Se considera como hogar a la unidad social conformada por una persona o grupo de personas que residen habitualmente en la misma vivienda particular y que se asocian para compartir sus necesidades de alojamiento, alimentación y otras necesidades básicas para vivir. El hogar es el conjunto de personas que viven bajo el mismo techo y comparten al menos los gastos en alimentación. Una persona sola también puede formar un hogar.
\item[Inscripción Inicial:  ] Es la cantidad de alumnos y alumnas que se registran o inscriben al 31 de marzo de cada año o ciclo escolar al centro educativo, asistiendo regularmente a clases. Se expresa en cifras absolutas.
\item[Matrícula: ] Individuos registrados oficialmente en un programa educativo determinado, o en una etapa o módulo asociado con este, independientemente de la edad.
\item[Nacimiento:  ] Nacimiento vivo es la expulsión o extracción completa del cuerpo de su madre, independientemente de la duración del embarazo, de un producto de la concepción que, después de dicha separación, respire o dé cualquier otra señal de vida, como latidos del corazón, pulsaciones del cordón umbilical o movimientos efectivos de los músculos de contracción voluntaria. 
\item[Niveles de educación:  ] Un set ordenado de programas educativos en relación a grados de experiencias de aprendizaje y a los conocimientos, destrezas y competencias que un programa educativo se propone impartir. La clasificación internacional de niveles educativos (CINE) refleja el grado de complejidad y especialización de los contenidos de un programa, desde lo básico hasta lo complejo.
\item[Población económicamente activa (PEA):  ] Todas las personas de 15 años o más, que en la semana de referencia realizaron algún tipo de actividad económica, y las personas que estaban disponibles para trabajar y hacen gestiones para encontrar un trabajo. Se incluyen también las personas que durante la semana de referencia no buscaron trabajo activamente por razones de mercado pero estaban dispuestas a iniciar un trabajo de forma inmediata. 
\item[Población desempleada:  ] Personas de 15 años o más, que no estando ocupadas en la semana de referencia, están disponibles y buscaron activamente incorporarse a alguna actividad económica en el lapso del último mes.
\item[Población ocupada:  ] Personas de 15 años o más, que durante la semana de referencia hayan realizado durante una hora o un día, alguna actividad económica, trabajando en el período de referencia por un sueldo o salario en metálico o especie o ausentes temporalmente de su trabajo; sin interrumpir su vínculo laboral con la unidad económica o empresa que lo contrata, es decir con empleo pero sin trabajar. 
\item[Pueblo:  ] Se entiende por pueblo a la nación constituida por “el conjunto de personas de un mismo origen étnico, que generalmente hablan un mismo idioma y tienen una tradición común”. No tiene la connotación de “minoría” como sí la tiene el concepto de grupo étnico. Una parte de las ciencias sociales reconoce a los “grupos étnicos” como nacionalidades cuando no han llegado aún al reconocimiento como estado-nación y como pueblos a los que ya tienen Estado propio. Por ello, existen pueblos dependientes y pueblos soberanos. En el Derecho Internacional y en los Acuerdos de Paz, no se da una definición explícita y precisa del concepto de pueblo, pero se le asimila a nación.
\item[Pueblo Mestizo, Ladino o no Indígena:  ] En los últimos diez años se ha dado cierta búsqueda y reacomodo en la identidad de los miembros del Pueblo Ladino o no indígena, por las siguientes razones: no todos los ladinos aceptan dicha denominación prefiriendo autodenominarse mestizos o viceversa. Ya se constató que hay no indígenas que se denominan “blancos” y “criollos”, y que no gustan ser considerados ni como ladinos ni como mestizos; y hay mestizos que no quieren ser considerados como pueblo sino como “conglomerado no indígena”, hasta que no se establezca la diversidad étnica y cultural en el seno del grupo no indígena. Puede decirse entonces, que ahora, uno de los vacíos de información que hay en el campo estadístico es al reconocimiento de la diversidad étnica en el seno de los no indígenas.
\item[Pueblos Indígenas:  ] Un pueblo es considerado indígena por el hecho de descender de poblaciones que habitaban en el país o en una región geográfica a la que pertenece el país en la época de la conquista, de la colonización o del establecimiento de las actuales fronteras estatales y que, cualquiera que sea su situación jurídica, conservan sus propias instituciones sociales, económicas, culturales y políticas, o parte de ellas. Además, la conciencia de su identidad indígena o tribal deberá considerarse un criterio fundamental para determinar los grupos (OIT, Convenio 169 de Pueblos Indígenas y Tribales, Art. 1).
\item[Tasa:  ] Es una proporción matemática entre dos variables multiplicada regularmente por cien o por mil. El denominador representa el universo en el cual se manifiesta el fenómeno representado en el numerador. Tasa de retención, tasa de aprobación, etc.
\item[Tasa bruta de escolarización: ] Establece una relación entre la inscripción inicial total sin distinción de edad, y la población que, según los reglamentos nacionales, debería estar siendo atendida. Se calcula regularmente por cada cien estudiantes. Su periodicidad es anual.

Proporciona el porcentaje de alumnos de todas las edades que se encuentran inscritos en un Nivel Educativo, independientemente de la edad con relación a la población en la edad oficial para dicho nivel.
\item[Tasa neta de escolaridad:  ] Es la relación que existe entre la parte de la inscripción inicial que se encuentra en la edad escolar oficial; para Nivel Primario: Primaria de niños, la población corresponde a la franja de 7 a 12 años y la población en edad escolar de 7 a 12 años. Se calcula regularmente por cada cien estudiantes. La periodicidad es anual.
\item[Tasa de repitencia:  ] Es la relación que existe entre el número de repitentes (t+1) y el número de alumnos que en el año  estaban inscritos en el mismo grado. Se calcula regularmente por cada cien estudiantes. La periodicidad es anual.
\item[Tasa de sobre-edad:  ] Es la relación que existe entre la cantidad de alumnos y alumnas inscritos en los diferentes grados de un nivel educativo, con dos o más años de atraso escolar por encima de la edad correspondiente al grado de estudio. Se calcula regularmente por cada cien estudiantes. La periodicidad es anual. Proporciona el porcentaje de alumnos con un atraso de dos o más años por encima de la edad correspondiente al grado de estudio de un nivel educativo.

\end{description}